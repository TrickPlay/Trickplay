\documentclass[10pt]{article}
\usepackage{wasysym}

\title{Poker Dawgs Application Summary}
\author{Devin Abbott, Rex Fenley, Darren Yin}

\begin{document}
\maketitle
\section*{Overview}
The Poker Dawgs application allows a user to play poker against
computer opponents. Currently the user can play against up to 5 other
AI players. The target audience of the application used to be normal
poker players, but the target audience is now people who want to learn
how to play. As such, we are adding a tutorial mode which will guide
players through a round of hold 'em before they start playing for
real.

\section*{Implementation}
The system's menus follow a general MVC model. The model is
initialized, and then the views and controllers are initialized on top
of the model. Every time the model is updated, \verb^model:notify()^
is called, and that in turn calls \verb^observer:update()^ on all
attached observers (aka all controllers and views in our
implementation).

The application begins by activating the
\verb^CharacterSelectionController^, which handles picking
players. The first player is assumed to be the human player, and every
player thereafter is a computer player.

When the user navigates to ``Start'' and hits Enter/OK on his remote,
the game begins. The model passes control to \verb^GameControl^ and
gets out of the way. Thereafter, \verb^GameControl^ is for the most
part self-sufficient. The entire game engine is designed with an
emphasis on event-driven architecture on top of an action
pipeline. Every time an event happens, we trigger the
\verb^GameControl^ \verb^on_event^ event listener, which runs the
action at the front of the pipeline. Depending on the return value of
the method call, it will either remove the action from the pipeline,
or not. Currently, every stage of the pipeline either starts a timer
event, or turns the keyboard listener on, but other events may be
added as necessary as well.

The \verb^GameControl^ has \verb^GamePresentation^ and
\verb^GameState^ slaves which it directs, while it also calls the
lower-level \verb^HandControl^ which has its own presentation and
state slaves.

Each betting round is composed of player turns, and for each turn,
there's a bet setup and a bet execution. The bet setup sets up a
listener for the event that a bet is ready (either through the
keyboard input interface, or from a timer after a computer has
gone). Then the bet execution receives the bet event and triggers a
call to HandState, which changes the state of the hand. The human
interface utilizes the \verb^BettingController^ (and its corresponding
\verb^BettingView^ gives it visibility). The computer move comes from
\verb^player:get_move(state)^.

\section*{Todo}
\begin{itemize}
\item Add side pot functionality
\item Make tutorial slide show more complete (add last few screens).
\item Wipe the dialogue bubbles after every round
\end{itemize}

\section*{Done}
\begin{itemize}
\item \CheckedBox Implement split pot functionality, so that if the
  best five-card hand is held simultaneously by two or more players,
  the pot is split between all the winners. If the pot doesn't split
  evenly (as will often be the case), then a random subset of players
  will get an extra dollar (unit of currency). Currently, if there's
  more than one winner, we just take the first player to be the
  winner.
\item \CheckedBox Right now, action for the next betting round is
  calculated wrong. It takes the dealer (index of dealer in players)
  and takes it to be the index of dealer in \verb^in_players^. It's
  right a lot of the time, and right now the user feedback from the
  system is so weak that you can't tell there's an issue here, but
  it'll be obvious when the initial action in certain betting rounds
  is wrong.
\item \CheckedBox Sanitize betting input so that it fails hard when
  the user tries to make an illegal bet.
\item \CheckedBox Remove a player when he loses all his money.
\item \CheckedBox Abstracted out initial endowment to a global
  variable
\item \CheckedBox Fixed a bug in Player where computer player doesn't
  actually go all in when he wishes to.
\item \CheckedBox Clean up the source files of all the extra cruft
  that doesn't actually run.
\item \CheckedBox Fix the bug where chips sometimes remain on the table.
\item \CheckedBox Make folding happen slower so that the human player
  can see and comprehend it easier.
\end{itemize}
\end{document}
